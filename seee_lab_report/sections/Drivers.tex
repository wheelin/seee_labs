\newpage
\section{Drivers}
\subsection{Environnement Qemu et plate-forme Reptar }
Lors de cette étape, nous allons déployer l'environnement de la cible à partir d'une image de carte MMC
(sdcard) et nous nous familiariserons avec l'insertion dynamique de module.
Le projet concerné est le projet drivers (répertoire du même nom à la racine du workspace). \\\\
\textbf{a) Donnée: }Lancez make dans le répertoire drivers/\\\\
\textbf{Travail réalisé: }
\begin{lstlisting}
redsuser@vm-reds-2015s2:~$ cd seee_student/drivers/
redsuser@vm-reds-2015s2:~/seee_student/drivers$ make
...
make[1]: Leaving directory `/home/redsuser/seee_student/linux-3.0-reptar'
arm-linux-gnueabihf-gcc -marm -I../linux-3.0-reptar -static buttons_test.c -o buttons_test
\end{lstlisting}
\textbf{b) Donnée: }A la racine du workspace, lancez les scripts suivants, puis la commande boot dans U-boot :
\begin{lstlisting}
$ ./deploy
$ ./stf
Reptar # boot 
\end{lstlisting}
\textbf{Travail réalisé: }
\begin{lstlisting}
redsuser@vm-reds-2015s2:~/seee_student/drivers$ cd ..
redsuser@vm-reds-2015s2:~/seee_student$ ./deploy 
Deploying into reptar rootfs ...
Mounting filesystem/sd-card.img...
[sudo] password for redsuser: 
SD card partitions mounted in 'boot_tmp' and 'filesystem_tmp' directories
Unmounting SD card image...
Synchronizing .img file
Unmounting 'boot_tmp' and 'filesystem_tmp'...
Done !
redsuser@vm-reds-2015s2:~/seee_student$ ./stf
...
Reptar # boot
reading uImage
...
*** Welcome on REPTAR (HEIG-VD/REDS): use root/root to log in ***
reptar login: root
Password: 
# 
\end{lstlisting}
\textbf{c) Donnée: }A la racine du rootfs (cd /), insérez le module avec la commande suivante :
\begin{lstlisting}
# insmod sp6.ko 
\end{lstlisting}
Vérifiez qu'il n'y ait aucun message d'erreur. La liste des modules chargés dynamiquement est obtenue
avec la commande lsmod et le retrait du module avec la commande rmmod 
\begin{lstlisting}
# lsmod
# rmmod sp6
reptar_sp6: bye bye!
# 
\end{lstlisting}
\textbf{Travail réalisé: }Avec la commande lsmod, on peut vérifier que notre module est correctement chargé. Si on le retire, il n'apparaît plus dans la liste.
\begin{lstlisting}
# cd /
# pwd
/
# insmod sp6.ko
reptar_sp6: module starting...
Probing FPGA driver (device: fpga)
input: reptar_sp6_buttons as /devices/platform/fpga/reptar_sp6_buttons/input/input1
reptar_sp6: done.
# lsmod
Module                  Size  Used by    Not tainted
sp6                     4606  0 
# rmmod sp6
reptar_sp6: bye bye!
# lsmod
Module                  Size  Used by    Not tainted
# 
\end{lstlisting}
\subsection{Driver de type caractère}
Cette étape consiste à travailler sur un driver de type caractère au niveau FPGA. Le code de cette
partie se trouve dans les fichiers reptar\_sp6.h et reptar\_sp6.c.
Sur la base du code existant, on souhaite pouvoir écrire et lire une chaîne de caractères contenant la
version du bitstream (hypothétique) dans la FPGA, stockée dans la variable globale bitstream\_version.\\\\
\textbf{a) Donnée: }Complétez les callbacks read() et write() afin qu'une application utilisateur puisse lire et écrire une
chaîne de (80 max.) caractères. \\\\
\textbf{Travail réalisé: }\\\\
\textbf{b) Donnée: }Pour identifier le nom de l'entrée dans /dev/ qui sera créée automatiquement, examinez la fonction
probe( ) du driver.\\\\
\textbf{Réponse aux questions: }
\begin{enumerate}
	\item Comment l'entrée dans /dev est-elle générée ? 
	\item  Quel sera le nom de l'entrée dans /dev ? \\
\end{enumerate}
\textbf{Travail réalisé: }\\\\
\textbf{c) Donnée: }Afin de tester votre driver, écrivez une application usertest (fichier usertest.c) qui écrira puis relira
la chaîne de version en utilisant l’entrée dans /dev évoquée ci-dessus.\\\\
\textbf{Réponse aux questions: }
\begin{enumerate}
	\item Recherchez les valeurs du major et minor attribuées à ce driver. Expliquez votre démarche\\
\end{enumerate}
\textbf{Travail réalisé: }\\\\
\subsection{Pilotage des LEDs }
Le code de pilotage des LEDs se trouve dans le fichier reptar\_sp6\_leds.c. La réalisation du driver des
LEDs.
\begin{enumerate}
	\item L’application graphique qtemu sera utilisée pour l'environnement émulé.
	\item Le driver devra être également testé sur la plate-forme réelle.
\end{enumerate}
Pour le pilotage des LEDs, on souhaite utiliser le sous-système leds présent dans le noyau Linux.
Effectuez un mappage du registre des LEDs à l'aide de la fonction ioremap( ), en vous servant de la
structure fpga\_resource.\\\\
\textbf{a) Donnée: }Enregistrez le device comme un device de type leds à l'aide de la fonction led\_classdev\_register( ).\\\\
\textbf{Travail réalisé: }\\\\
\textbf{b) Donnée: }Cherchez et implémentez le(s) callback(s) gérant l'enclenchement/déclenchement des LEDs.\\\\
\textbf{Travail réalisé: }\\\\
\textbf{Réponse aux questions: }
\begin{enumerate}
	\item Combien y a-t-il de devices de type LED gérés par notre driver ?\\
\end{enumerate}
\textbf{c) Donnée: }Testez le driver LED dans l'environnement qtemu (application graphique). Lancez le script
ledstest.sh.\\\\
\textbf{Travail réalisé: }\\\\
\textbf{d) Donnée: }Testez votre driver sur la plate-forme (réelle) Reptar.\\\\
\textbf{Travail réalisé: }\\\\
\subsection{Pilotage des boutons}
Lors de cette étape, nous travaillerons sur le driver gérant la pression des boutons. L'objectif est de
contrôler une application dans l'espace utilisateur à l'aide des boutons.\\\\
\textbf{a) Donnée: }Complétez la fonction probe( ) pour l'enregistrement des deux callbacks d'interruption (traitement
immédiat + traitement différé) à l'aide de la fonction request\_threaded\_irq().\\\\
\textbf{Travail réalisé: }\\\\
\textbf{b) Donnée: }Implémentez le traitement immédiat lié à l'interruption. Il devra :
\begin{enumerate}
	\item stocker la valeur du registre bouton dans le champ current\_button de la structure privée
	(l'adresse du registre contenant cette information est également disponible dans la structure
	privée)
	\item acquitter l'interruption.\\
\end{enumerate}
\textbf{Travail réalisé: }\\\\
\textbf{c) Donnée: }Testez votre driver boutons à l'aide de l'application buttons\_test. Dans l'environnement émulé,
l'application devra ouvrir le fichier /dev/input/event1. Il faudra taper la commande suivante:
\begin{lstlisting}
# ./buttons_test -e1
\end{lstlisting}
Testez les boutons un par un.\\\\
\textbf{Travail réalisé: }\\\\