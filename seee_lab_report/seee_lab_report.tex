\documentclass[a4paper,12pt]{article}
\usepackage[french]{babel} 
\usepackage[T1]{fontenc}
%\usepackage[ansinew]{inputenc}
\usepackage[utf8]{inputenc}
\usepackage[top=3cm, bottom=3cm, left=2.3cm,right=2cm]{geometry}
\usepackage{graphicx}
\usepackage{color}
\usepackage{listings}
%\usepackage{marvosym}
%\usepackage{yfonts}
\usepackage[normalem]{ulem}
\usepackage{verbatim}
\usepackage{listings}
\usepackage{float}
%\renewcommand{\thesection}{\arabic{section}}
\usepackage{array} % pour les tableaux
\usepackage{amsmath} % pour les équations
\usepackage{float}
\usepackage{hyperref}	% crée des liens dans le pdf
\hypersetup{					% colorise les liens du pdf
  colorlinks=true,
  urlcolor=black
	citecolor=black,
  linkcolor=black,
  urlcolor=blue
}
\usepackage{url}			% change la police des url (utilisation : \url{http://asdf.ch})
\definecolor{dkgreen}{rgb}{0,0.6,0}
\definecolor{gray}{rgb}{0.5,0.5,0.5}
\definecolor{mauve}{rgb}{0.58,0.01,0.82}
%[babel=true]
\usepackage{csquotes}
\lstset{ %
  language=C,                % the language of the code
  basicstyle=\footnotesize,           % the size of the fonts that are used for the code
  numbers=left,                   % where to put the line-numbers
  numberstyle=\tiny\color{gray},  % the style that is used for the line-numbers
  stepnumber=1,                   % the step between two line-numbers. If it's 1, each line 
                                  % will be numbered
  numbersep=5pt,                  % how far the line-numbers are from the code
  backgroundcolor=\color{white},      % choose the background color. You must add \usepackage{color}
  showspaces=false,               % show spaces adding particular underscores
  showstringspaces=false,         % underline spaces within strings
  showtabs=false,                 % show tabs within strings adding particular underscores
  frame=single,                   % adds a frame around the code
  rulecolor=\color{black},        % if not set, the frame-color may be changed on line-breaks within not-black text (e.g. commens (green here))
  tabsize=2,                      % sets default tabsize to 2 spaces
  captionpos=b,                   % sets the caption-position to bottom
  breaklines=true,                % sets automatic line breaking
  breakatwhitespace=false,        % sets if automatic breaks should only happen at whitespace
  title=\lstname,                   % show the filename of files included with \lstinputlisting;
                                  % also try caption instead of title
  keywordstyle=\color{blue},          % keyword style
  commentstyle=\color{dkgreen},       % comment style
  stringstyle=\color{mauve},         % string literal style
  escapeinside={\%*}{*)},            % if you want to add a comment within your code
  morekeywords={*,...}               % if you want to add more keywords to the set
}

%Table of content with dot
\usepackage{etoolbox}
\makeatletter
\patchcmd{\l@section}
{\hfil}
{\leaders\hbox{\normalfont$\m@th\mkern \@dotsep mu\hbox{.}\mkern \@dotsep mu$}\hfill}
{}{}
\makeatother

%no indentation
\setlength{\parindent}{0cm}

%en-tête
\usepackage{fancyhdr}
\lhead{SEEE}
\chead{}
\rhead{\today}
\pagestyle{fancy}

% Title Page
\title{\Huge{\textsc{Systèmes d'exploitation et environnements d'exécution embarqués}} \\ 
	\Huge{\textbf{Rapport de laboratoire}} \\
	\huge{Master HES-SO}}
\author{Émilie \textsc{Gsponer}, Grégory \textsc{Emery} }
\date{\today \\
	version 1.0}

%-------------------------début du document-------------------------------------
\begin{document}

\maketitle % page de garde
\newpage
\tableofcontents % table des matières
\newpage
\section{Introduction REPTAR}
\subsection{Mise en place de l'environnement, utilisation de git}
\textbf{a) Donnée: }Il faut tout d'abord récupérer le dépôt étudiant pour les laboratoires SEEE à l'aide de la commande
suivante (via une fenêtre de terminal):
\begin{lstlisting}
$ git clone firstname.lastname@eigit.heig-vd.ch:/home2/reds/seee/seee_student
\end{lstlisting}
\textbf{Travail réalisé: }
Nous n'avions pas les droits d'accès pour le dépôt git, nous l'avons donc téléchargé, puis extrait depuis le lien:
\url{https://drive.switch.ch/index.php/s/TbHxQZtmO9IVdkb}.\\
Le dossier seee\_student a ensuite été placé dans: /home/redsuser/\\

\textbf{b) Donnée: }Lancez Eclipse et ouvrez le workspace seee\_student. Vous devriez obtenir la liste des projets (à
gauche). Chaque projet a un lien symbolique dans la racine du workspace. \\\\
\textbf{Travail réalisé: }
En introduisant le path du dossier seee\_student comme workspace au lancement d'Eclipse, nous obtenons la liste de projets suivante:
\begin{figure}[H]
	\begin{center}
		\includegraphics[height=7cm]{img/eclipseProjet.png}
		\caption{Liste des projets}
		\label{eclipseProjet}
	\end{center}
\end{figure}
\textbf{c) Donnée: }Compilez maintenant l'émulateur Qemu. Dans une fenêtre de terminal, lancez la commande
suivante à partir de votre répertoire seee\_student : 
\begin{lstlisting}
$ make qemu
\end{lstlisting}
\textbf{Travail réalisé: }
Vu que nous n'avons pas téléchargé le dossier de projets depuis git, il faut nettoyer le contenu du dossier avec clean ou distclean avant de pourvoir utiliser qemu. Le make qemu prend quelques instants.\\
\begin{lstlisting}
redsuser@vm-reds-2015s2:~/seee_student$ make clean
redsuser@vm-reds-2015s2:~/seee_student$ make qemu
...
make[1]: Leaving directory `/home/redsuser/seee_student/qemu-reds'
redsuser@vm-reds-2015s2:~/seee_student$
\end{lstlisting}
\subsection{Démarrage de Qemu}
\textbf{a) Donnée: }Depuis Eclipse, lancez le debugger avec la configuration de debug « qemu-reds Debug ». Dans la
fenêtre Console, vous pourrez entrer directement des commandes de U-boot (tapez help par
exemple). \\\\
\textbf{Travail réalisé: }
\begin{figure}[H]
	\begin{center}
		\includegraphics[height=7cm]{img/manip2Intro.png}
		\caption{Lancement d'Eclipse en mode Debug}
		\label{eclipseDebug}
	\end{center}
\end{figure}
\textbf{Remarque: }Après le lancement du Debug, il faut changer d'onglet en haut à droite en choisissant \textit{Debug} pour avoir la console. Ce changement d'onglet ne se fait pas automatiquement.
\begin{figure}[H]
	\begin{center}
		\includegraphics[width=18cm]{img/ubootCommand.png}
		\caption{Command help dans l'U-boot}
		\label{ubootCommand}
	\end{center}
\end{figure}
\textbf{b) Donnée: } Interrompez l'exécution du programme en cliquant sur l'icône pause. Identifiez la ligne en cours d'exécution dans le code source. \\\\
\textbf{Travail réalisé: }En interrompant le programme avec le bouton \textit{suspend}, on obtient la vue assembleur ci-dessous. L'environnement essaie d'ouvrir le fichier ppoll.c, on est donc en attente d'un événement. Le programme est interrompu après un syscall.
\begin{figure}[H]
	\begin{center}
		\includegraphics[width=18cm]{img/ubootAsm.png}
		\caption{Command help dans l'U-boot}
		\label{ubootAsm}
	\end{center}
\end{figure}
\textbf{c) Donnée: } Stoppez l'exécution, et dans une fenêtre de commande, démarrez qemu à l'aide du script stf (en
tapant ./stf) dans le répertoire racine. Vous arrivez dans U-boot. \\\\
\textbf{Travail réalisé: }Cette partie n'a plus rien avoir avec Eclipse, on peut le fermer et lancer un terminal.
Avec la commande \textit{stf} tapée à la racine du répertoire seee\_student, on arrive au même point qu'en lançant le Debug dans Eclipse. On peut également essayer la commande \textit{help}
\begin{lstlisting}
redsuser@vm-reds-2015s2:~/seee_student$ ./stf
WARNING: Image format was not specified for 'filesystem/flash' and probing guessed raw.
...

Reptar # help
?       - alias for 'help'
base    - print or set address offset
bdinfo  - print Board Info structure
boot    - boot default, i.e., run 'bootcmd'
bootd   - boot default, i.e., run 'bootcmd'
bootm   - boot application image from memory
bootp   - boot image via network usi
...
\end{lstlisting}
\subsection{Tests avec U-boot}
\textbf{a) Donnée: }Dans U-boot, listez les variables d'environnement avec la commande printenv. Observez les
variables prédéfinies « tftp1, tftp2 et goapp ». Ces variables définissent des commandes U-boot qui
peuvent être exécutées à l'aide de la commande run (par exemple run tftp1).
La commande go <addr> permet de lancer l'exécution à l'adresse physique <addr>.
Vous pouvez définir/modifier vos propres variables et les sauvegarder dans la flash émulée avec la
commande saveenv (seulement avec le lancement via stf). \\\\
\textbf{Travail Réalisé: }Après être entré dans l'U-boot avec \textit{stf}, nous avons pu lister les variables d'environnement suivantes:
\begin{lstlisting}
redsuser@vm-reds-2015s2:~$ cd seee_student/
redsuser@vm-reds-2015s2:~/seee_student$ ./stf
WARNING: Image format was not specified for 'filesystem/flash' and probing guessed raw.
...
goapp=go 0x81600000
...
tftp1=tftp helloworld_u-boot/helloworld.bin
tftp2=tftp gpio_u-boot/gpio_u-boot.bin

Environment size: 930/4092 bytes
Reptar # 
\end{lstlisting}
Les variables \textit{tftp1} et \textit{tftp2} sont des alias permettant de lancer des applications, la variable goapp est un alias permettant de lancer l'exécution de l'adresse physique 0x81600000. Elle définit l'adresse de début des applications. Voici un exemple d'utilisation de ces variables:
\begin{lstlisting}
redsuser@vm-reds-2015s2:~$ cd seee_student/
redsuser@vm-reds-2015s2:~/seee_student$ cd helloworld_u-boot/
redsuser@vm-reds-2015s2:~/seee_student/helloworld_u-boot$ make
...
redsuser@vm-reds-2015s2:~/seee_student/helloworld_u-boot$ cd ../gpio_u-boot/
redsuser@vm-reds-2015s2:~/seee_student/gpio_u-boot$ make
...
redsuser@vm-reds-2015s2:~/seee_student/gpio_u-boot$ cd ..
redsuser@vm-reds-2015s2:~/seee_student$ ./stf 
WARNING: Image format was not specified for 'filesystem/flash' and probing guessed raw.
...
Reptar # run tftp1
smc911x: detected LAN9118 controller
smc911x: phy initialized
smc911x: MAC e4:af:a1:40:01:fe
Using smc911x-0 device
TFTP from server 10.0.2.2; our IP address is 10.0.2.10
Filename 'helloworld_u-boot/helloworld.bin'.
Load address: 0x81600000
Loading: #
done
Bytes transferred = 776 (308 hex)

Reptar # run goapp
## Starting application at 0x81600000 ...
Example expects ABI version 6
Actual U-Boot ABI version 6
Hello World
argc = 1
argv[0] = "0x81600000"
argv[1] = "<NULL>"
Hit any key to exit ... 
## Application terminated, rc = 0x0

Reptar # run tftp2
smc911x: detected LAN9118 controller
smc911x: phy initialized
smc911x: MAC e4:af:a1:40:01:fe
Using smc911x-0 device
TFTP from server 10.0.2.2; our IP address is 10.0.2.10
Filename 'gpio_u-boot/gpio_u-boot.bin'.
Load address: 0x81600000
Loading: #
done
Bytes transferred = 3080 (c08 hex)

Reptar # run goapp
## Starting application at 0x81600000 ...
Start of the GPIO U-boot Standalone Application
Stop of the GPIO U-boot Standalone Application
## Application terminated, rc = 0x0
Reptar #
\end{lstlisting}
La commande run tftp<x> charge une application à l'adresse 0x81600000, tandis que run goapp va exécuter l'application à cette adresse comme le montre l'exemple ci-dessus.\\\\
\textbf{b) Donnée: }La production de l'exécutable helloworld\_u-boot s'effectue en tapant la commande make dans le
répertoire contenant les sources du programme. Ensuite, vous pouvez transférer le fichier (extension
.bin) dans U-boot et exécuter le binaire (aidez-vous des variables d'environnement prédéfinies). \\\\
\textbf{Travail réalisé: } Ce point a été fait en même temps que le précédent.\\

\textbf{c) Donnée: } Testez le debugger dans Eclipse avec le projet helloworld\_u-boot. Mettez un breakpoint dans le
code source au démarrage du programme, et lancez le debugger avec la configuration de debug
« helloworld\_u-boot Debug ». \\\\
\textbf{Travail Réalisé: } Il faut que U-boot soit démarré dans un terminal externe avec \textit{stf} pour que la manipulation fonctionne avec Eclipse.
\begin{figure}[H]
	\begin{center}
		\includegraphics[width=18cm]{img/ubootCommand2.png}
		\caption{Debug d'hello\_world\_u-boot}
		\label{ubootComm2}
	\end{center}
\end{figure}
\textbf{Remarque: }\color{red}Qemu se comporte comme un serveur GDB, ce qui permet à Eclipse  de communiquer avec lui et de debugger des applications...à retravailler\color{black}
\subsection{Tests avec Linux}
\textbf{a) Donnée: }Lancez le script ./deploy qui permettra de déployer le noyau Linux dans la sdcard virtuelle (ignorez
l'erreur due à l'absence de certains fichiers). \\\\
\textbf{Travail réalisé: }
\begin{lstlisting}
redsuser@vm-reds-2015s2:~/seee_student$ ./deploy 
Deploying into reptar rootfs ...
Mounting filesystem/sd-card.img...
[sudo] password for redsuser: 
SD card partitions mounted in 'boot_tmp' and 'filesystem_tmp' directories
cp: cannot stat 'drivers/sp6.ko': No such file or directory
cp: cannot stat 'drivers/usertest': No such file or directory
cp: cannot stat 'drivers/buttons_test': No such file or directory
Unmounting SD card image...
Synchronizing .img file
Unmounting 'boot_tmp' and 'filesystem_tmp'...
Done !
redsuser@vm-reds-2015s2:~/seee_student$ 
\end{lstlisting}
\textbf{b) Donnée: }Poursuivez ensuite en cross-compilant l'application helloworld pour Linux (via make). \\\\
\textbf{Travail réalisé: }
\begin{lstlisting}
redsuser@vm-reds-2015s2:~/seee_student$ cd helloworld_linux/
redsuser@vm-reds-2015s2:~/seee_student/helloworld_linux$ make
...
redsuser@vm-reds-2015s2:~/seee_student/helloworld_linux$ cd ..
\end{lstlisting}
\textbf{c) Donnée: }Copiez l'exécutable dans le rootfs\\\\
\textbf{Travail réalisé: }
\begin{lstlisting}
redsuser@vm-reds-2015s2:~/seee_student$ ./mount-sd.sh 
Mounting filesystem/sd-card.img...
SD card partitions mounted in 'boot_tmp' and 'filesystem_tmp' directories

redsuser@vm-reds-2015s2:~/seee_student$ sudo cp helloworld_linux/helloworld filesystem_tmp/root

redsuser@vm-reds-2015s2:~/seee_student$ ./umount-sd.sh 
Unmounting SD card image...
Synchronizing .img file
Unmounting 'boot_tmp' and 'filesystem_tmp'...
Done !
redsuser@vm-reds-2015s2:~/seee_student$ 
\end{lstlisting}
\textbf{d) Donnée: }Lancez le script stq suivi de la commande boot dans U-boot pour amorcer le démarrage de Linux\\\\
\textbf{Travail réalisé: } Avec la commande stq, une représentation de la carte se lance.
\begin{lstlisting}
redsuser@vm-reds-2015s2:~/seee_student$ ./stq 
libGL error: failed to authenticate magic 1
libGL error: failed to load driver: vboxvideo
Running QEMU
...
Warning: smc911x-0 MAC addresses don't match:
Address in SROM is         52:54:00:12:34:56
Address in environment is  e4:af:a1:40:01:fe
Reptar # boot
reading uImage
...
*** Welcome on REPTAR (HEIG-VD/REDS): use root/root to log in ***
reptar login: root
Password: 
# 
\end{lstlisting}
\begin{figure}[H]
	\begin{center}
		\includegraphics[width=18cm]{img/linux.png}
		\caption{Environnement émulé}
		\label{linux}
	\end{center}
\end{figure}
\textbf{e) Donnée: }Lancez votre application\\\\
\textbf{Travail réalisé: }
\begin{lstlisting}
# ls
Settings         fs               helloworld       rootfs_domU.img
# ./helloworld 
Hello world within Linux
argv[0] = ./helloworld
# 
\end{lstlisting}
\textbf{f) Donnée: }Dans Linux, tapez la commande suivante :
\begin{lstlisting}
$ /usr/share/qt/examples/effects/lighting/lighting -qws & 
\end{lstlisting}
\textbf{Travail réalisé: }Cette commande permet de lancer une application pré installée de l'émulateur.
\begin{figure}[H]
	\begin{center}
		\includegraphics[width=10cm]{img/linux2.png}
		\caption{Lancement d'une application}
		\label{linux2}
	\end{center}
\end{figure}
\subsection{Tests sur la plate-forme réelle}
\textbf{a) Donnée: }Déployez l'application helloworld dans U-boot sur la plate-forme REPTAR avec l'interface réseau.
Le transfert peut s'effectuer avec la commande tftp.
Il est nécessaire d’exécuter la commande suivante pour mettre à jour les adresses IP et MAC de la plate-forme
REPTAR : 
\begin{lstlisting}
# run setmac setip 
\end{lstlisting}
\textbf{Travail réalisé: }Avec la commande tftp il faut donner comme paramètre le \textit{.bin} de l'application ainsi que l'adresse physique où charger le programme. Cette adresse est 0x81600000 comme dans les exercices précédents.
\begin{lstlisting}
redsuser@vm-reds-2015s2:~/seee_student$ cd helloworld_u-boot/
redsuser@vm-reds-2015s2:~/seee_student/helloworld_u-boot$ make
redsuser@vm-reds-2015s2:~/seee_student/helloworld_u-boot$ cp helloworld.bin /home/redsuser/tftpboot
redsuser@vm-reds-2015s2:~/seee_student/helloworld_u-boot$ sudo picocom -b 115200 /dev/ttyUSB0 
[sudo] password for redsuser: 
picocom v1.7
...
Terminal ready

Reptar # run setmac setip
Reptar # tftp 0x81600000 helloworld.bin
smc911x: detected LAN9220 controller
smc911x: phy initialized
smc911x: MAC e4:af:a1:40:01:fe
Using smc911x-0 device
TFTP from server 192.168.1.1; our IP address is 192.168.1.254
Filename 'helloworld.bin'.
Load address: 0x81600000
Loading: T #
done
Bytes transferred = 776 (308 hex)

Reptar # go 0x81600000
## Starting application at 0x81600000 ...
Example expects ABI version 6
Actual U-Boot ABI version 6
Hello World
argc = 1
argv[0] = "0x81600000"
argv[1] = "<NULL>"
Hit any key to exit ... 
\end{lstlisting}
\textbf{Remarque: }La commande tftp ne fonctionnera pas tant que la configuration réseau n'est pas correcte. Il faut impérativement que l'adresse Ip de la connexion par pont de la VM soit 192.168.1.1.
\begin{figure}[H]
	\begin{center}
		\includegraphics[width=10cm]{img/ipConfig.png}
		\caption{Configuration réseau}
		\label{ipConfig}
	\end{center}
\end{figure}
\textbf{b) Donnée: }Déployez l'application helloworld dans Linux à l'aide du réseau et de la commande scp.\\\\
\textbf{Travail réalisé: } Nous avons découvert que l'adresse Ip de la carte n'était pas celle attendue, nous avons donc dû adapter scp pour l'adresse Ip 192.168.1.254.
\begin{lstlisting}
Reptar # boot
reading uImage
...
*** Welcome on REPTAR (HEIG-VD/REDS): use root/root to log in ***
reptar login: root
Password: 
# ifconfig
eth0      Link encap:Ethernet  HWaddr E4:AF:A1:40:01:FE  
inet addr:192.168.1.254  Bcast:192.168.1.255  Mask:255.255.255.0
...
\end{lstlisting}
La commande scp permet de transférer le helloworld\_linux à la carte reptar par l'interface réseau depuis la machine hôte.
\begin{lstlisting}
redsuser@vm-reds-2015s2:~/seee_student/helloworld_linux$ scp helloworld root@192.168.1.254:helloworld

The authenticity of host '192.168.1.254 (192.168.1.254)' can't be established.
RSA key fingerprint is fb:59:a3:73:97:9d:b7:b9:8a:40:e8:bc:19:ab:ab:70.
Are you sure you want to continue connecting (yes/no)? yes
Warning: Permanently added '192.168.1.254' (RSA) to the list of known hosts.
root@192.168.1.254's password: 
helloworld                                    100% 6877     6.7KB/s   00:00    
\end{lstlisting}
L'application helloworld est maintenant chargée sur la cible, il ne reste plus qu'à l'exécuter sur celle-ci.
\begin{lstlisting}
# ls
bitstreams  helloworld     tests
# ./helloworld 
Hello world within Linux
argv[0] = ./helloworld
#
\end{lstlisting}
\subsection{Accès aux périphériques REPTAR}
\textbf{a) Donnée: } Sur la base de l’exemple gpio\_u-boot., vous devez développer une application permettant d’interagir avec les
LEDs et les switchs présents sur la carte CPU de la plate-forme REPTAR.
Le but de l’application est d’allumer une LED lorsqu’on appuie sur un switch.
\begin{enumerate}
	\item La LED 0 doit s’allumer lorsqu’on appuie sur le SWITCH 0.
	\item La LED 1 s’allume si l’on appuie sur le SWITCH 1.
	\item Et ainsi de suite pour les LEDs et switchs 0..3 de la carte CPU.
\end{enumerate}
Le switch numéro 4 sert à quitter l’application. Aidez-vous des fichiers d'en-tête (\#include) déjà présents dans
le chablon fourni.\\
L’application gpio\_u-boot est à déployer dans U-boot via la commande tftp. \\\\
\textbf{Emplacement du code réalisé: }/gpio\_u-boot.c\\\\
Ce code est très basique, mais implémente correctement les points exigés par la donnée. Les commandes suivantes ont permis de lancer l'application sur la cible réelle dans l'U-boot. Une pression sur le switch numéro 4 permet de terminer l'application.\\
\begin{lstlisting}
redsuser@vm-reds-2015s2:~/seee_student/gpio_u-boot$ make
...
redsuser@vm-reds-2015s2:~/seee_student/gpio_u-boot$ cp gpio_u-boot.bin /home/redsuser/tftpboot
redsuser@vm-reds-2015s2:~/seee_student/gpio_u-boot$ sudo picocom -b 115200 /dev/ttyUSB0 
[sudo] password for redsuser: 
picocom v1.7
...
Terminal ready

Reptar # tftp 0x81600000 gpio_u-boot.bin
smc911x: detected LAN9220 controller
smc911x: phy initialized
smc911x: MAC e4:af:a1:40:01:fe
Using smc911x-0 device
TFTP from server 192.168.1.1; our IP address is 192.168.1.254
Filename 'gpio_u-boot.bin'.
Load address: 0x81600000
Loading: T #
done
Bytes transferred = 776 (308 hex)

Reptar # go 0x81600000
...
Stop of the GPIO U-boot Standalone Application
Reptar #
\end{lstlisting}
\newpage
\section{Émulation de périphériques}
\subsection{Environnement Qemu et machine Reptar}
Cette étape vous permet de vous familiariser avec l’environnement que nous utiliserons pour
l’émulation de périphériques.
Dans cette étape, il est nécessaire de travailler avec l'application graphique Qtemu, qui constitue le
frontend graphique de Qemu. L'application est développée en C++ et utilise la librairie Qt. \\\\
\textbf{a) Donnée: }A partir du répertoire seee\_student, lancez le frontend graphique avec le script stq \\\\
\textbf{Travail réalisé: }
\begin{lstlisting}
$ cd ~/seee_student/
$ ./stq
...
Reptar # 
\end{lstlisting}
\begin{figure}[H]
	\begin{center}
		\includegraphics[width=10cm]{img/emulation1.png}
		\caption{Frontend graphique de Qemu}
		\label{emulation1}
	\end{center}
\end{figure}
\textbf{b) Donnée: }Les fichiers sources de Qemu se trouvent dans le répertoire qemu-reds. Examinez les fichiers
suivants :
\begin{enumerate}
	\item hw/arm/reptar/reptar.c Emulation plate-forme REPTAR
	\item hw/reptar\_sp6.c Emulation de la FPGA
	\item hw/reptar\_sp6\_clcd.c Emulation gestion du LCD4x20
	\item hw/reptar\_sp6\_buttons.c Emulation gestion des boutons
	\item hw/reptar\_sp6\_emul.c Gateway entre Qemu et Qtemu
\end{enumerate}
Vous trouverez également toute la documentation nécessaire sur la plate-forme Reptar dans le
répertoire doc. \\\\
\textbf{Remarque: }Ces différents fichiers implémentent ce qui ressemble à des modules noyaux.\\\\
\textbf{c) Donnée: } La compilation de Qemu pourra s'effectuer dans le répertoire qemu-reds directement, avec la
commande make (utilisez make -j4 ou -j8 pour aller plus vite !). \\\\
\textbf{Travail réalisé: }Par la suite, seule la commande \textit{make} sera nécessaire pour recompiler l'émulateur.\\
En lançant \textit{./qtemu} et Eclipse, on pourra debugger l'émulateur Qemu.
\begin{lstlisting}
$ cd ~/seee_student/qemu-reds/
$ ./configure --target-list=arm-softmmu --enable-debug --disable-attr --disable-docs 
...
$ make -j8
...
$
\end{lstlisting}
\subsection{Émulation de la FPGA Spartan6}
Dans cette étape, il s'agit de mettre en place la structure nécessaire à l'émulation de la FPGA intégrée
à la plate-forme. La FPGA implémente des registres associés aux périphériques externes. Dans cette
étape, il s'agit de s'assurer que l'accès aux adresses I/O en lecture et écriture fonctionne. \\\\
\textbf{a) Donnée: }Complétez l'émulation de la FPGA afin de tester l'écriture et la lecture à l'une ou l'autre adresse
dédiée à la FPGA (affichez simplement un message). \\\\
\textbf{Emplacement du code:}\\\textit{/emulationSpartan6\_part2/reptar\_sp6.c}\\ \textit{/emulationSpartan6\_part2/reptar.c}\\\\
\textbf{Travail réalisé: }Nous avons modifié les fichiers \textit{reptar\_sp6.c} et \textit{reptar.c}\\
Le point crucial de cette partie du labo a été de trouvé l'adresse de base de la FPGA qui est \textbf{0x18000000}. \underline{Cette information a été trouvée dans la documentation, plus précisément le}\\\underline{ fichier Spartan6\_Registers\_SEEE\_2014\_22mar14.xlsx}. Tous les autres offsets et adresses données dans la suite du document proviennent de ce document. Nous avons en effet besoin de cette adresse pour lier/créer le \textit{reptar\_sp6} dans la partie reptar. 
\begin{lstlisting}
	sysbus_create_simple("reptar_sp6",0x18000000,NULL);
\end{lstlisting}
Pour le reste de l'implémentation, nous nous somme basé sur le diagramme de séquence du support de cours et avons pris le document \textit{versatilepb.c} comme exemple pour le contenu des méthodes callback.\\
Le bon fonctionnement du code a été "testé" premièrement en réussissant la compilation sans erreurs, puis le lancement sans crash. Nous avons également ajouté des messages affichés dans la console dans les différentes méthodes callback pour suivre l'initialisation. L'exécution a été faite de la manière suivante:
\begin{lstlisting}
$ cd ~/seee_student/qemu-reds/
$ make
...
$ cd ..
$ ./stq
...
sp6 init
reptar-sp6-emul: sp6_emul_init
sp6 initfn
...
Reptar # 
\end{lstlisting}
\textbf{b) Donnée: }Testez les accès en lecture-écriture avec U-Boot. \\\\
\textbf{Travail réalisé: } Pour l'instant, les callback de lecture/écriture que nous avons implémentés contiennent uniquement des messages d'indication qui sont affichés dans la console. À l'aide des commandes suivantes, nous avons pu tester leur bon fonctionnement. Les commandes suivantes tentent de lire, puis écrire à l'adresse de base de la FPGA.
\begin{lstlisting}
Reptar # md.l 0x18000000 1
18000000:sp6 read
 00000000    ....
Reptar # mw.l 0x18000000 1
sp6 write
Reptar #
\end{lstlisting}
\subsection{Émulation des devices de type LED (output)}
\textbf{Donnée: }La FPGA est connectée à des LEDs qui sont visibles sur l'interface graphique. Cette étape consiste à
implémenter le code d'émulation précédent afin de gérer l'accès aux LEDs reliées à la FPGA.
Les interactions entre la FPGA et l'interface graphique doivent être gérées proprement. \\\\
\textbf{Emplacement du code:}\textit{/emulationSpartan6\_part3/reptar\_sp6.c}\\\\
\textbf{Travail réalisé: }Pour cette partie, il a fallu rechercher l'offset du registre des LEDs qui est \textit{0x003A}. Il faut donc ajouter cet offset à l'adresse de base de la FPGA.\\ Nous avons défini une variable qui garde la valeur écrite dans le registre des LEDs pour permettre la relecture de la valeur.\\ Pour la lecture et l'écriture, on teste si l'offset correspond et si l'on lit/écrit des données de la bonne taille, soit 16bits. Des messages ont été implémentés pour indiquer si la lecture/écriture est faite correctement.
\\La valeur lue est simplement affichée dans la console. Pour l'écriture, la valeur écrite est transmise à l'aide d'un message JSON conformément aux directives du \textit{Guide d'utilisation de l'infrastructure}.
\\Nous avons testé le bon fonctionnement du code avec le test 10 de itbok\footnote{itbok, signifie Is The Board OK}. Le test montre que l'on arrive à lire et écrire correctement sur les leds.
\begin{figure}[H]
	\begin{center}
		\includegraphics[width=15cm]{img/leds1.png}
		\caption{Test d'allumage des LEDs}
		\label{leds1}
	\end{center}
\end{figure}
\begin{figure}[H]
	\begin{center}
		\includegraphics[width=15cm]{img/leds2.png}
		\caption{Test d'extinction des LEDs}
		\label{leds2}
	\end{center}
\end{figure}
\subsection{Émulation de type boutons (input)}
La FPGA est connectée à une série de boutons (switches) sur la plate-forme Reptar. Cette étape consiste
à mettre en place la structure nécessaire à la gestion de ces boutons. \\\\
\textbf{a) Donnée: }Adaptez les fichiers nécessaires afin que l'émulation de votre périphérique (FPGA) puisse détecter
la pression d'une touche, sans vous préoccuper pour le moment des interruptions. \\\\
\textbf{Emplacement du code:}\\\textit{/emulationSpartan6\_part4/reptar\_sp6.c}\\
\textit{/emulationSpartan6\_part4/reptar\_sp6\_buttons.c}\\\\
\textbf{Travail réalisé: }Le code de cette partie est inspiré du \textit{Guide d'utilisation de l'infrastructure}. L'offset pour lire la valeur des boutons est \textit{0x0012}. Lorsqu'un bouton est pressé, le handler du fichier sp6\_button est appelé et la valeur du registre est mémorisée. La valeur des boutons peut également être récupérée lors d'une lecture du registre correspondant.\\\\
\textbf{b) Donnée: }Le projet sp6\_buttons\_u-boot contient une application permettant de tester vos boutons (en mode
polling). Compilez l'application et effectuez quelques tests.\\\\
\textbf{Travail réalisé: }L'application a été compilée avec make. Il faut ensuite lancer l'émulateur avec la commande \textit{stq}. Une fois dans l'U-boot, on peut lancer l'application testant les boutons. Elle est enregistrée dans les variables d'environnements sous le nom \textit{tftp3}. Les lignes ci-dessous démontrent le bon fonctionnement des boutons. Le bouton exit arrête l'application. 
\begin{lstlisting}
$ cd ~/seee_student
$ ./stq
...
Reptar # run tftp3
smc911x: detected LAN9118 controller
smc911x: phy initialized
smc911x: MAC e4:af:a1:40:01:fe
Using smc911x-0 device
TFTP from server 10.0.2.2; our IP address is 10.0.2.10
Filename 'sp6_buttons_u-boot/sp6_buttons.bin'.
Load address: 0x81600000
Loading: #######
done
Bytes transferred = 34512 (86d0 hex)

Reptar # run goapp
## Starting application at 0x81600000 ...
Start of the SP6 buttons standalone test application.
Button ONE pressed
...
Button ONE pressed
Button ONE pressed
Button ONE pressed
reptar-sp6-emul: sp6_emul_event_handle: read 29 
reptar-sp6-emul: sp6_emul_event_handle: cJSON_Parse done 
Button status : 0x0

Button LEFT pressed
Button LEFT pressed
Button LEFT pressed
reptar-sp6-emul: sp6_emul_event_handle: read 29 
reptar-sp6-emul: sp6_emul_event_handle: cJSON_Parse done 
Button status : 0x0

reptar-sp6-emul: sp6_emul_event_handle: read 30 
reptar-sp6-emul: sp6_emul_event_handle: cJSON_Parse done 
Button status : 0x10
Button EXIT pressed
SP6 buttons standalone test application exit.
## Application terminated, rc = 0x0
Reptar # reptar-sp6-emul: sp6_emul_event_handle: read 29 
reptar-sp6-emul: sp6_emul_event_handle: cJSON_Parse done 
Button status : 0x0
\end{lstlisting}
\subsection{Gestion des interruptions (IRQ) avec les boutons}
Complétez votre émulateur avec le code nécessaire à la gestion d'une interruption à niveau émise par
la FPGA lorsqu'un bouton est pressé. L'interruption est censée être acquittée par le driver. Il faut donc
gérer l'état interne associé à cette interruption. \\\\
\textbf{a) Donnée: }Commencez par adapter le code d'initialisation de la plate-forme (reptar.c) afin d'instancier une
interruption en provenance de la FPGA ; l'interruption sera de type niveau.\\\\
\textbf{Emplacement du code:}\\\textit{/emulationSpartan6\_part5/reptar\_sp6.c}\\
\textit{/emulationSpartan6\_part5/reptar.c}\\
\textit{/emulationSpartan6\_part5/reptar\_sp6\_buttons.c}\\\\
\textbf{Travail réalisé: }La première étape a consisté à assigner le reptar\_sp6 sur la pin GPIO 10 dans le fichier \textit{reptar.c}.\\Il a ensuite fallu faire en sorte de générer une interruption de type niveau lors d'une pression sur un bouton. Cela a été fait dans le fichier \textit{reptar\_sp6\_buttons.c}. Notre code génère l'interruption uniquement si les IRQ ont été préalablement autorisées en configurant les registres de contrôle. Nous avons choisi de ne pas générer d'interruptions lors de la relâche du bouton (0x0).\\
Finalement, le fichier \textit{reptar\_sp6.c} a été adapté pour permettre la lecture et l'écriture du registre d'IRQ des boutons. Lorsque le registre est lu, on va lire la valeur des bouton et l'ajouter dans le registre de status de l'IRQ, puis retourner le tout. Pour l'écriture, un masquage est fait afin de savoir si l'on veut activer et/ou quittancer les interruptions et dans ce cas repasser la GPIO10 à l'état bas.\\\\
\textbf{b) Donnée: }Testez que l'interruption fonctionne en configurant le contrôleur GPIO et en interrogeant le registre
d'état, dans U-Boot. Les registres du microcontrôleur à utiliser sont les suivants :
GPIO\_RISINGDETECT, GPIO\_IRQENABLE1 et GPIO\_IRQSTATUS1
De plus, l'interruption doit aussi être activée au niveau de la FPGA (cf documentation). \\\\
\textbf{Travail réalisé: }Notre code a dans un premier temps été testé à l'aide d'\textit{itbok} avec le test numéro 30. Cela a permis de valider le bon fonctionnement des interruptions avec tous les boutons. On peut voir que le message \textit{IRQ RAISE} ne s'affiche plus si l'on désactive les interruptions.
\begin{lstlisting}
	Press on SW7...
	reptar-sp6-emul: sp6_emul_event_handle: read 29 
	reptar-sp6-emul: sp6_emul_event_handle: cJSON_Parse done 
	Button status : 0x0
	reptar-sp6-emul: sp6_emul_event_handle: read 30 
	reptar-sp6-emul: sp6_emul_event_handle: cJSON_Parse done 
	Button status : 0x40
	IRQ RAISE
	sp6_read: Button irq status read 0x8d (button value 0x7)
	sp6_write: Button irq status write 0x81
	Enable IRQ
	Clear IRQ
	IRQ detected:
	- button: 7  ......... Test PASSED
	Press on SW8...
	reptar-sp6-emul: sp6_emul_event_handle: read 29 
	reptar-sp6-emul: sp6_emul_event_handle: cJSON_Parse done 
	Button status : 0x0
	reptar-sp6-emul: sp6_emul_event_handle: read 31 
	reptar-sp6-emul: sp6_emul_event_handle: cJSON_Parse done 
	Button status : 0x80
	IRQ RAISE
	sp6_read: Button irq status read 0x8f (button value 0x8)
	sp6_write: Button irq status write 0x81
	Enable IRQ
	Clear IRQ
	IRQ detected:
	- button: 8  ......... Test PASSED
	sp6_write: Button irq status write 0x1
	Disable IRQ
	Clear IRQ
	
	IRQ test complete. Press Enter to exit
	reptar-sp6-emul: sp6_emul_event_handle: read 29 
	reptar-sp6-emul: sp6_emul_event_handle: cJSON_Parse done 
	Button status : 0x0
	reptar-sp6-emul: sp6_emul_event_handle: read 31 
	reptar-sp6-emul: sp6_emul_event_handle: cJSON_Parse done 
	Button status : 0x80
	reptar-sp6-emul: sp6_emul_event_handle: read 29 
	reptar-sp6-emul: sp6_emul_event_handle: cJSON_Parse done 
	Button status : 0x0
\end{lstlisting}
Notre code a ensuite été testé avec les registres GPIO du microcontrôleur (\textit{GPIO\_RISINGDETECT, GPIO\_IRQENABLE1, GPIO\_IRQSTATUS1}). Il faut utiliser pour ce faire le banque GPIO1, ce qui donne les adresses de registre correspondantes : 0x48310048, 0x4831001C et 0x48310018. Comme les boutons sont sur la GPIO 10, il faut aller autoriser l'interruption en activant le bit 10 des registres GPIO\_RISINGDETECT et  GPIO\_IRQENABLE1. Il ne faut pas oublier d'aller ensuite autoriser les interruptions sur la FPGA. La capture ci-dessous présente cette configuration. Après chaque interruption, il faut aller quittancer le bit 10 du GPIO\_IRQSTATUS1 et également intervenir au niveau de la FPGA.
\begin{lstlisting}
#Enable interrupts in GPIO
Reptar # mw.l 0x4831001C 0x00000400        
Reptar # mw.l 0x48310048 0x00000400  

#Enable interrupts in FPGA
Reptar # mw.w 0x18000018 0x0081
sp6_write: Button irq status write 0x81
Enable IRQ
Clear IRQ

#The IRQ are now available, the status register doesn't have detect interrupt yet
md.l 0x48310018 1
48310018: 00000000    ....

#Generate an interrupt
Reptar # reptar-sp6-emul: sp6_emul_event_handle: read 30 
reptar-sp6-emul: sp6_emul_event_handle: cJSON_Parse done 
Button status : 0x40
IRQ RAISE
reptar-sp6-emul: sp6_emul_event_handle: read 29 
reptar-sp6-emul: sp6_emul_event_handle: cJSON_Parse done 
Button status : 0x0

#Check status register, interrupt has been detected
md.l 0x48310018 1
48310018: 00000400    ....

#Need to ack the interrupt
Reptar # mw.l 0x48310018 0x00000400
Reptar # mw.w 0x18000018 0x0081
sp6_write: Button irq status write 0x81
Enable IRQ
Clear IRQ

#...Repeat operations
\end{lstlisting}
\subsection{Émulation de l'afficheur 7 segments}
La FPGA est connectée à un afficheur 7 segments, visible sur l’émulateur. Cette étape consiste à mettre
en place la gestion de cet afficheur 7 segments.\\\\
\textbf{a) Donnée: }Adaptez les fichiers nécessaires afin que l'émulation de votre périphérique (FPGA) puisse gérer les
trois digits de l’afficheur 7 segments. \\\\
\textbf{Emplacement du code:}\textit{/emulationSpartan6\_part6/reptar\_sp6.c}\\\\
\textbf{Travail réalisé: }Pour cette partie, nous avons simplement ajouté le code pour écrire et lire dans le registre de chacun des trois digits de l'affichage 7 segments. La valeur de chaque affichage est stocké dans une variable. Pour l'écriture, il faut spécifier dans le message Json le nom du périphérique, le numéro du digit ainsi que la valeur à afficher.\\\\
\textbf{b) Donnée: }Le dossier 7seg\_u-boot contient une application permettant de tester l’afficheur 7 segments : les
chiffres de 0 à 9 doivent défiler progressivement : 012, puis 123, 234, 456, 567, …, 901, 012, etc.
Compilez l'application et effectuez quelques tests.\\\\
\textbf{Travail réalisé: }Une fois l'application compilée, il a fallu la lancer dans l'U-boot. Comme elle n'est pas définie dans les variables d'environnement, il faut utiliser la commande complète. Une fois lancée, l'application va incrémenter la valeur des digits. L'image ci-dessous démontre le bon fonctionnement.
\begin{lstlisting}
$ cd ~/seee_student
$ ./stq
...
Reptar # tftp 7seg_u-boot/7seg_u-boot.bin
smc911x: detected LAN9118 controller
smc911x: phy initialized
smc911x: MAC e4:af:a1:40:01:fe
Using smc911x-0 device
TFTP from server 10.0.2.2; our IP address is 10.0.2.10
Filename '7seg_u-boot/7seg_u-boot.bin'.
Load address: 0x81600000
Loading: #######
done
Bytes transferred = 34932 (8874 hex)
Reptar # run goapp
\end{lstlisting}
\begin{figure}[H]
	\begin{center}
		\includegraphics[width=15cm]{img/emulation2.png}
		\caption{Frontend graphique de Qemu avec affichage 7 segments}
		\label{emulation2}
	\end{center}
\end{figure}
\textbf{Modifications: }\\ Suite à l'examen intermédiaire, nous avons remarqué que nous n'avons pas initialisé correctement l'émulation de la carte Reptar. Nous avons modifié la fonction \textit{sp6\_initfn} du fichier \textit{reptar\_sp6.c} comme ci-dessous. Notre erreur était de passer l'adresse de base du sp6 dans la fonction memory\_region\_init\_io alors que le paramètre demandé était la taille mémoire pour notre périphérique. 
\begin{lstlisting}
//memory_region_init_io(&s->iomem, OBJECT(s), &reptar_sp6_ops, s,
//                  "reptar_sp6", SP6_BASE_ADDRESS);

//Allocate 1MB for reptar
memory_region_init_io(&s->iomem, OBJECT(s), &reptar_sp6_ops, s,
"reptar_sp6", 0x100000);
\end{lstlisting}
Nous avons donc modifié le code pour allouer la bonne taille, soit 1MB, en accord avec la documentation de la plateforme. L'erreur n'a pas eu de conséquence sur les tests, on a juste alloué beaucoup plus de mémoire que nécessaire.\\\\
\textbf{Emplacement du code modifié: }\textit{/emulationSpartan6\_part6/reptar\_sp6\_modifs.c}
\subsection{Mini-application utilisant les boutons et l'afficheur 7 segments}
\textbf{a) Donnée: }Le dossier miniapp\_u-boot contient un chablon. Complétez-le afin de créer une application qui
utilise les boutons SW2, SW5, SW4 et SW3, ainsi que l’afficheur 7 segments.
\begin{enumerate}
	\item Lors d’un appui sur SW2, SW5 ou SW4, le digit respectivement à gauche, au centre ou au
	milieu est incrémenté de 1, modulo 10. Si un digit atteint 9, il reviendra à 0. 
	\item La valeur initiale de chaque digit, au démarrage de l’application, est 0 (on affichera 000). 
	\item Un appui sur SW3 quitte l’application. 
	\item Vous devrez gérer l’anti-rebond : le digit ne devra être incrémenté que si le bouton est
	pressé puis relâché (comme un appui sur une touche de sonnette par exemple). \\
\end{enumerate}
\textbf{Emplacement du code: }\textit{/miniapp\_test\_emulation/miniapp\_u-boot.c}\\\\
\textbf{Travail réalisé: }L'application a été codée avec une boucle \textit{while} exécutant de manière répétitive les étapes suivantes:
\begin{itemize}
	\item On lit le registre d'état des boutons et on stocke sa valeur dans une première variable. 
	\item On attend un petit moment.
	\item On relit le registre d'état des boutons et on stocke cette nouvelle valeur dans une seconde variable.
	\item Ensuite, on teste les deux variables avec des masques pour savoir si :
	\begin{itemize}
		\item Dans la première variable, le masquage retourne une valeur supérieure à 0 et indique que le bouton testé est enfoncé.
		\item Dans la seconde variable, le masquage retourne une valeur égale à 0 et indique que le bouton a été relâché. 
	\end{itemize}
	Ces tests sont effectués pour les trois boutons d'incrémentation des 7 segments et pour le bouton d'arrêt de l'application. 
	\item Si un des tests passe :
	\begin{itemize}
		\item Si c'est un des boutons d'incrémentation, on appelle la fonction \textit{incr\_7seg} avec en paramètre l'index du 7 segments à incrémenter.
		\item Si c'est le bouton d'arrêt, on quitte la boucle avec un \textit{break}.\\
	\end{itemize}
\end{itemize}
Nous avons remarqué une chose lors de la compilation. Il s'avère que le compilateur prend comme point d'entrée la première instruction du programme, avec le Makefile fourni. Lors des premiers essais, la fonction \textit{incr\_7seg} était implémentée avant le \textit{main}, ce qui en faisait le point d'entrée du programme. Après correction, c'est à dire, définition du prototype de la fonction avant le \textit{main} et implémentation après, tout fonctionne correctement.\\\\
\textbf{b) Donnée: }Testez votre application sur l’émulateur\\\\
\textbf{Travail réalisé: }\\
La figure \ref{emul_miniapp_init} montre l'application qui vient de démarrer. Tous les 7-segments sont initialisés à zero.
\begin{figure}[H]
	\begin{center}
		\includegraphics[width=16cm]{img/emul_test_miniapp_init.png}
		\caption{Initialisation des 7-segments durant le lancement de l'application sur le frontend d'émulation}
		\label{emul_miniapp_init}
	\end{center}
\end{figure}
La figure \ref{emul_miniapp_counting} montre l'application avec plusieurs 7-segments incrémentés.
\begin{figure}[H]
	\begin{center}
		\includegraphics[width=16cm]{img/emul_test_miniapp_counting.png}
		\caption{Incrémentation des 7-segments du frontend d'émulation}
		\label{emul_miniapp_counting}
	\end{center}
\end{figure}
La figure \ref{emul_miniapp_end} montre la fin de l'application. Le bouton tout en bas de la croix a été pressé pour l'arrêter. 
\begin{figure}[H]
	\begin{center}
		\includegraphics[width=16cm]{img/emul_test_miniapp_end.png}
		\caption{Frontend graphique de Qemu avec affichage 7 segments}
		\label{emul_miniapp_end}
	\end{center}
\end{figure}

\textbf{c) Donnée: }Déployez et testez votre application sur la plate-forme réelle\\\\
\textbf{Travail réalisé: }Une fois l'application compilée, il a fallu copier le \textit{.bin} dans le dossier \textit{tftpboot}. On peut ensuite lancer l'application depuis l'U-boot en accédant la carte Reptar avec \textit{picocom}. La pression sur le switch 3 réussit ici aussi à terminer l'application.
\begin{lstlisting}
$ cd ~/seee_student
$ sudo picocom -b 115200 /dev/ttyUSB0
[sudo] password for redsuser: 
...
Terminal ready

Reptar # tftp 0x81600000 miniapp_u-boot.bin
smc911x: detected LAN9220 controller
smc911x: phy initialized
smc911x: MAC e4:af:a1:40:01:0a
Using smc911x-0 device
TFTP from server 192.168.1.1; our IP address is 192.168.1.10
Filename 'miniapp_u-boot.bin'.
Load address: 0x81600000
Loading: T ###
done
Bytes transferred = 35292 (89dc hex)

Reptar # go 0x81600000
## Starting application at 0x81600000 ...
Start of the Miniapp U-boot Standalone Application
Stop of the Miniapp U-boot Standalone Application
## Application terminated, rc = 0x0
Reptar # 
\end{lstlisting}

Et ici, une photo de l'application en train de tourner sur la plateforme:
\begin{figure}[H]
    \begin{center}
        \includegraphics[width=15cm]{img/miniapp_reptar.JPG}
        \caption{Programme miniapp en fonctionnement sur la plateforme physique}
        \label{miniapp_reptar}
    \end{center}
\end{figure}

\newpage
\section{Drivers}
\subsection{Environnement Qemu et plate-forme Reptar }
Lors de cette étape, nous allons déployer l'environnement de la cible à partir d'une image de carte MMC
(sdcard) et nous nous familiariserons avec l'insertion dynamique de module.
Le projet concerné est le projet drivers (répertoire du même nom à la racine du workspace). \\\\
\textbf{a) Donnée: }Lancez make dans le répertoire drivers/\\\\
\textbf{Travail réalisé: }
\begin{lstlisting}
$ cd ~/seee_student/drivers/
$ make
...
make[1]: Leaving directory `/home/redsuser/seee_student/linux-3.0-reptar'
arm-linux-gnueabihf-gcc -marm -I../linux-3.0-reptar -static buttons_test.c -o buttons_test
$
\end{lstlisting}
\textbf{b) Donnée: }A la racine du workspace, lancez les scripts suivants, puis la commande boot dans U-boot :
\begin{lstlisting}
$ ./deploy
$ ./stf
Reptar # boot 
\end{lstlisting}
\textbf{Travail réalisé: }
\begin{lstlisting}
$ cd ~/seee_student/drivers
$ ./deploy 
Deploying into reptar rootfs ...
Mounting filesystem/sd-card.img...
[sudo] password for redsuser: 
SD card partitions mounted in 'boot_tmp' and 'filesystem_tmp' directories
Unmounting SD card image...
Synchronizing .img file
Unmounting 'boot_tmp' and 'filesystem_tmp'...
Done !
$ ./stf
...
Reptar # boot
reading uImage
...
*** Welcome on REPTAR (HEIG-VD/REDS): use root/root to log in ***
reptar login: root
Password: 
# 
\end{lstlisting}
\textbf{c) Donnée: }A la racine du rootfs (cd /), insérez le module avec la commande suivante :
\begin{lstlisting}
# insmod sp6.ko 
\end{lstlisting}
Vérifiez qu'il n'y ait aucun message d'erreur. La liste des modules chargés dynamiquement est obtenue
avec la commande lsmod et le retrait du module avec la commande rmmod 
\begin{lstlisting}
# lsmod
# rmmod sp6
reptar_sp6: bye bye!
# 
\end{lstlisting}
\textbf{Travail réalisé: }Avec la commande lsmod, on peut vérifier que notre module est correctement chargé. Si on le retire, il n'apparaît plus dans la liste.
\begin{lstlisting}
# cd /
# pwd
/
# insmod sp6.ko
reptar_sp6: module starting...
Probing FPGA driver (device: fpga)
input: reptar_sp6_buttons as /devices/platform/fpga/reptar_sp6_buttons/input/input1
reptar_sp6: done.
# lsmod
Module                  Size  Used by    Not tainted
sp6                     4606  0 
# rmmod sp6
reptar_sp6: bye bye!
# lsmod
Module                  Size  Used by    Not tainted
# 
\end{lstlisting}
\subsection{Driver de type caractère}
Cette étape consiste à travailler sur un driver de type caractère au niveau FPGA. Le code de cette
partie se trouve dans les fichiers reptar\_sp6.h et reptar\_sp6.c.
Sur la base du code existant, on souhaite pouvoir écrire et lire une chaîne de caractères contenant la
version du bitstream (hypothétique) dans la FPGA, stockée dans la variable globale bitstream\_version.\\\\
\textbf{a) Donnée: }Complétez les callbacks read() et write() afin qu'une application utilisateur puisse lire et écrire une
chaîne de (80 max.) caractères. \\\\
\textbf{Emplacement du code:}\textit{ /Drivers/char\_driver\_ex1/reptar\_sp6.c}\\\\
\textbf{Travail réalisé: }Les callbacks read et write ont été implémentés très sommairement dans le module avec la fonction 
\textit{copy\_from/to\_user}. Il faut faire attention à bien retourner le nombre de bytes qui ont été lus ou écrits, pour que l'application puisse être au fait et détecter le cas échéant, si une erreur s'est produite.\\

\begin{lstlisting}
	ssize_t fpga_read(struct file *filp, char *buffer, size_t length, loff_t *offset) {
		
		/* If offset is not zero, it means that a previous read already occured.
		* So, we tell the user space that we are at the end
		*/
		
		if (*offset != 0)
			return 0;
		if(copy_to_user(buffer, bitstream_version, sizeof(bitstream_version)) != 0)
		{
			return -EFAULT;
		}
		return sizeof(bitstream_version);			
	}
	
	ssize_t fpga_write(struct file *filp, const char *buff, 
		size_t len, loff_t *off) {
		if(len > 80)
			return -EFAULT;
		
		if(copy_from_user(bitstream_version, buff, len) != 0)
			return -EFAULT;
		return len;
	}
\end{lstlisting}

\begin{lstlisting}
# cd /
# insmod sp6.ko 
reptar_sp6: module starting...
Probing FPGA driver (device: fpga)
input: reptar_sp6_buttons as /devices/platform/fpga/reptar_sp6_buttons/input/input1
reptar_sp6: done.
# ./usertest 
Device ID : 0
Inode number : 5
Protection mode : 0
Num of hard links : 680
User ID of owner : 8624
Group ID of owner : 1
Device ID (spec files only): 0
Total size [bytes] : 0
Setting bitstream version to : mais coucou mon petit                                                          
Device ID : 0
Inode number : 5
Protection mode : 0
Num of hard links : 680
User ID of owner : 8624
Group ID of owner : 1
Device ID (spec files only): 0
Total size [bytes] : 0
Normally read buffer
Bitstream version : mais coucou mon petit
\end{lstlisting}

\textbf{b) Donnée: }Pour identifier le nom de l'entrée dans /dev/ qui sera créée automatiquement, examinez la fonction
probe( ) du driver.\\\\
\textbf{Emplacement du code: }\textit{/Drivers/char\_driver\_ex1/reptar\_sp6.c}\\\\
\textbf{Réponse aux questions: }
\begin{enumerate}
	\item Comment l'entrée dans /dev est-elle générée ? 
	\begin{lstlisting}
		fpga_cdev = cdev_alloc();
		
		/* We store the newly allocated cdev as a private driver data to this device */
		dev_set_drvdata(&pdev->dev, fpga_cdev);
		
		/* Initializing character device to enable user space ops */
		cdev_init(fpga_cdev, &fpga_fops);
		cdev_add(fpga_cdev, pdev->dev.devt, 1);
		device_create(pdata->fpga_class, NULL, pdev->dev.devt, NULL, "sp6");
	\end{lstlisting}
	\item  Quel sera le nom de l'entrée dans /dev ? \\
	Le nom sera \textit{sp6}, le string passé en argument à la fonction \textit{device\_create}.
	\begin{lstlisting}
		ls /dev | grep sp 
	\end{lstlisting}
	La commande ci-dessus donne en réponse \textit{sp6}.
\end{enumerate}
\textbf{c) Donnée: }Afin de tester votre driver, écrivez une application usertest (fichier usertest.c) qui écrira puis relira
la chaîne de version en utilisant l’entrée dans /dev évoquée ci-dessus.\\

\textbf{Emplacement du code : }\textit{/Drivers/char\_driver\_ex1/usertest.c}\\

\textbf{Travail réalisé: }Une application a aussi été codée pour tester les callbacks. Celle-ci affiche les stats du fichier créé dans le dossier \textit{/dev}, écrit une version de bitstream totalement fabuliste, relit les stats et enfin lit le bitstream pour confirmer le succès de l'écriture.\\

Output du code \textit{usertest}:
\begin{lstlisting}
	# ./usertest 
	Device ID : 0
	Inode number : 5
	Protection mode : 0
	Num of hard links : 680
	User ID of owner : 8624
	Group ID of owner : 1
	Device ID (spec files only): 0
	Total size [bytes] : 0
	Setting bitstream version to : mais coucou mon petit                                                          
	Device ID : 0
	Inode number : 5
	Protection mode : 0
	Num of hard links : 680
	User ID of owner : 8624
	Group ID of owner : 1
	Device ID (spec files only): 0
	Total size [bytes] : 0
	Bitstream version : mais coucou mon petit  
\end{lstlisting}
\textbf{Réponse aux questions: }
\begin{enumerate}
	\item Recherchez les valeurs du major et minor attribuées à ce driver. Expliquez votre démarche\\
\end{enumerate}
La démarche est la suivante:
\begin{lstlisting}
	# ls -l /dev/ | grep sp6
	crw-rw----    1 root     root      252,   0 Jun 11 11:31 sp6
\end{lstlisting}
Le \textit{major} est \textit{252} et le \textit{minor} est \textit{0}. Le \textit{0} indique que c'est la seule instance du driver en fonctionnement.


\subsection{Pilotage des LEDs }
Le code de pilotage des LEDs se trouve dans le fichier reptar\_sp6\_leds.c. La réalisation du driver des
LEDs.
\begin{enumerate}
	\item L’application graphique qtemu sera utilisée pour l'environnement émulé.
	\item Le driver devra être également testé sur la plate-forme réelle.
\end{enumerate}
Pour le pilotage des LEDs, on souhaite utiliser le sous-système leds présent dans le noyau Linux.
Effectuez un mappage du registre des LEDs à l'aide de la fonction ioremap( ), en vous servant de la
structure fpga\_resource.\\\\
\textbf{a) Donnée: }Enregistrez le device comme un device de type leds à l'aide de la fonction led\_classdev\_register( ).\\\\
\textbf{Emplacement du code: }\textit{/Drivers/pilotageLeds/reptar\_sp6\_leds.c}\\\\
\textbf{Travail réalisé: }Nous avons modifié le fichier \textit{reptar\_sp6\_leds.c}. Avant de pouvoir enregistrer le device, il faut mapper le registre des LEDs avec la fonction \textit{ioremap}. Puis, on peut enregistrer le device comme ci-dessous:
\begin{lstlisting}
/* Map the LED register */
led->reg = ioremap(fpga_resource->start+pdata->reg_offset,2);

/* Register our new led device into led class */
led_classdev_register(&pdev->dev.parent, &led->cdev);
\end{lstlisting}
\textbf{b) Donnée: }Cherchez et implémentez le(s) callback(s) gérant l'enclenchement/déclenchement des LEDs.\\\\
\textbf{Travail réalisé: }En regardant dans la fonction \textit{probe}, on voit que le callback pour enregistrer les LEDs a été lié avec la fonction \textit{reptar\_sp6\_led\_set}.
\begin{lstlisting}
led->cdev.brightness_set = reptar_sp6_led_set;
led->cdev.name = pdata->name;
\end{lstlisting}
Pour interagir avec l'état des LEDs, il nous suffit de modifier la valeur du registre que l'on a précédent mapper avec \textit{ioremap}. On accède directement au périphérique. La modification du registre est protégée par un \textit{spin\_lock} pour gérer l'accès concurrent, car le registre est partagé.
\begin{lstlisting}
/* Protect access to shared register */
spin_lock(&reg_lock);

/* to be completed */
if(value)
*led->reg |= 0x0001 << (pd->bit);
else
*led->reg &= ~ (0x0001 << (pd->bit));
spin_unlock(&reg_lock);
\end{lstlisting}
\textbf{Réponse aux questions: }
\begin{enumerate}
	\item Combien y a-t-il de devices de type LED gérés par notre driver ?\\
\end{enumerate}
Il y a 6 LEDs selon la déclaration dans reptar\_sp6.h et il y a bien 6 LEDs dans la structure reptar\_sp6\_leds\_pdata[] du fichier reptar\_sp6.h
\begin{lstlisting}
/* Only LEDS 0 to 5 are under CPU control. 6 and 7 are used by the FPGA itself */
#define SP6_NUM_LEDS 6
\end{lstlisting}
\textbf{c) Donnée: }Testez le driver LED dans l'environnement qtemu (application graphique). Lancez le script
ledstest.sh.\\\\
\textbf{Travail réalisé: }Pour tester le driver dans l'émulateur, il ne faut plus travailler dans l'U-boot, mais dans Linux. Pour cela, il faut entrer la commande \textit{boot} au démarrage de l'émulateur. Le script \textit{ledstest.sh} peut être lancé une fois le driver inséré dans le noyau avec la commande \textit{insmod}. On peut ensuite suivre la séquence d'allumage/extinction des LEDs dans la représentation graphique \textit{qtemu}.
\begin{lstlisting}
$ cd ~/seee_student/drivers/
$ make
...
make[1]: Leaving directory /home/redsuser/seee_student/linux-3.0-reptar
$ cd ..
$ ./deploy
...
Done !
$ ./stq
...

Reptar # boot
...
*** Welcome on REPTAR (HEIG-VD/REDS): use root/root to log in ***
reptar login: root
Password: 

# cd /
# insmod sp6.ko 
reptar_sp6: module starting...
Probing FPGA driver (device: fpga)
input: reptar_sp6_buttons as /devices/platform/fpga/reptar_sp6_buttons/input/input1
reptar_sp6: done.

# ./ledstest.sh 
SP6 leds present in sysfs !
Turning on LEDS from right to left...
sp6_read: Led read 0x0
sp6_write: Led write 0x1
reptar-sp6-emul: sp6_emul_cmd_post
reptar-sp6-emul: sp6_emul_cmd_post Inserting into queue...
reptar-sp6-emul: sp6_emul_cmd_post ...done
sp6_read: Led read 0x1
sp6_write: Led write 0x3
reptar-sp6-emul: sp6_emul_cmd_post
reptar-sp6-emul: sp6_emul_cmd_post Inserting into queue...
reptar-sp6-emul: sp6_emul_cmd_post ...done
sp6_read: Led read 0x3
...
End of test
\end{lstlisting}
L'image ci-dessous montre que l'on arrive à piloter les LEDs
\begin{figure}[H]
	\begin{center}
		\includegraphics[width=17cm]{img/driverLeds.png}
		\caption{Test du pilote des LEDs avec qtemu}
		\label{driverLed3}
	\end{center}
\end{figure}
\textbf{d) Donnée: }Testez votre driver sur la plate-forme (réelle) Reptar.\\\\
\textbf{Travail réalisé: }Pour le test sur la plateforme réelle, nous avons également besoin de travailler dans Linux, on doit donc insérer la carte SD où se trouve l'image du noyau dans la plateforme, sans quoi on ne pourra pas booter. Nous avons remarqué que l'adresse IP de la carte Reptar avait encore changé. Maintenant c'est l'adresse 192.168.1.10.
\begin{lstlisting}
$ sudo picocom -b 115200 /dev/ttyUSB0 
[sudo] password for redsuser: 
picocom v1.7
...
Terminal ready

Reptar # boot
...
*** Welcome on REPTAR (HEIG-VD/REDS): use root/root to log in ***
reptar login: root
Password: 
# ifconfig
eth0      Link encap:Ethernet  HWaddr E4:AF:A1:40:01:0A  
inet addr:192.168.1.10  Bcast:192.168.1.255  Mask:255.255.255.0
...
\end{lstlisting}
Depuis la machine hôte, on va transférer par la connexion réseau les fichiers ledstest.sh et sp6.ko nécessaire. Nous avons besoin pour cela de la commande scp et de l'adresse IP de la plateforme
\begin{lstlisting}
$ cd ~/seee_student/drivers/
$ scp ledstest.sh root@192.168.1.10:ledstest.sh
root@192.168.1.10's password: 
ledstest.sh                                   100%  772     0.8KB/s   00:00  
  
$ scp sp6.ko root@192.168.1.10:sp6.ko
root@192.168.1.10's password: 
sp6.ko                                        100%  161KB 161.1KB/s   00:00    
\end{lstlisting}
Une fois les fichiers copiés, on accède à la plateforme par la connexion série et l'on peut installer notre driver et lancer le script de tests.
\begin{lstlisting}
# ls
bitstreams   helloworld   ledstest.sh  sp6.ko       tests
# insmod sp6.ko 
reptar_sp6: module starting...
Probing FPGA driver (device: fpga)
input: reptar_sp6_buttons as /devices/platform/fpga/reptar_sp6_buttons/input/input2
reptar_sp6: done.
# ./ledstest.sh 
SP6 leds present in sysfs !
Turning on LEDS from right to left...
Turning off LEDS from right to left...
All LEDS ON...
All LEDS OFF...
End of test
# rmmod sp6.ko 
reptar_sp6: bye bye!
\end{lstlisting}
Les images ci-dessous prouvent le bon fonctionnement de notre pilote des LEDs. On voit qu'elles s'allument et s'éteignent.
\begin{figure}[H]
	\begin{center}
		\includegraphics[width=17cm]{img/driverLeds3.png}
		\caption{Allumage des LEDs}
		\label{driverLed2}
	\end{center}
\end{figure}
\begin{figure}[H]
	\begin{center}
		\includegraphics[width=17cm]{img/driverLeds2.png}
		\caption{Extinction des LEDs}
		\label{driverLed}
	\end{center}
\end{figure}
\subsection{Pilotage des boutons}
Lors de cette étape, nous travaillerons sur le driver gérant la pression des boutons. L'objectif est de
contrôler une application dans l'espace utilisateur à l'aide des boutons.\\\\
\textbf{a) Donnée: }Complétez la fonction probe( ) pour l'enregistrement des deux callbacks d'interruption (traitement
immédiat + traitement différé) à l'aide de la fonction request\_threaded\_irq().\\\\
\textbf{Emplacement du code: }\textit{/Drivers/buttons\_driver/reptar\_sp6\_buttons.c}\\\\ 
\textbf{Travail réalisé: }
On utilise la fonction \textit{request\_threaded\_irq}, qui prend en paramètre :
\begin{itemize}
	\item Le numéro d'interruption
	\item Un pointeur vers une irq\_handler immédiate
	\item Un pointeur vers une irq\_handler threadée
	\item Les flags 
	\item Un pointeur vers les données qu'on veut passer aux irq\_handlers
\end{itemize} 
Puis on active les interruptions provenant de la FPGA.
\begin{lstlisting}
	if((ret = request_threaded_irq(btns->irq, reptar_sp6_buttons_irq, 
									reptar_sp6_buttons_irq_thread, 
									IRQF_TRIGGER_RISING, "THR_IRQ", btns))) {
		printk("Cannot request threaded irq for gpio : ret = %d\n", ret);
		return -EIO;
	}
	// enable fpga irq
	*(btns->irq_reg) |= 0x0080;
\end{lstlisting}

\textbf{b) Donnée: }Implémentez le traitement immédiat lié à l'interruption. Il devra :
\begin{enumerate}
	\item stocker la valeur du registre bouton dans le champ current\_button de la structure privée
	(l'adresse du registre contenant cette information est également disponible dans la structure
	privée)
	\item acquitter l'interruption.\\
\end{enumerate}
\textbf{Travail réalisé: }
La première chose qui est faite, est de récupérer l'état du registre des buttons de la FPGA. Depuis le registre, on récupère ensuite la source de l'interruption et le statut de l'interruption. Si il y a bien eu interruption et que la source est bien la pression sur un bouton, alors on copie la valeur du registre des boutons dans le champ \textit{current\_button} de la structure \textit{reptar\_sp6\_buttons}. 

\begin{lstlisting}
	/* Hard/immediate interrupt handler */
	static irqreturn_t reptar_sp6_buttons_irq(int irq, void *dev_id)
	{
		struct reptar_sp6_buttons *dev = (struct reptar_sp6_buttons *)dev_id;
		int irq_reg = *(dev->irq_reg);
		int irq_source = (irq_reg >> 5) & 0x03;
		int irq_status = (irq_reg >> 4) & 0x01;	
		
		if(irq_status) 
		{
			if(irq_source == 0) 
			{
				dev->current_button = *(dev->btns_reg);
			}
			*(dev->irq_reg) = *(dev->irq_reg) | 0x0001;
		}
		return IRQ_WAKE_THREAD;
	}
\end{lstlisting}

\textbf{c) Donnée: }Testez votre driver boutons à l'aide de l'application buttons\_test. Dans l'environnement émulé,
l'application devra ouvrir le fichier /dev/input/event1. Il faudra taper la commande suivante:
\begin{lstlisting}
# ./buttons_test -e1
\end{lstlisting}
Testez les boutons un par un.\\\\
\textbf{Travail réalisé: }
On lance la commande ci-dessus et l'on voit des lignes apparaitre avec le type, le code et le front du bouton pressé. 
\begin{figure}[H]
	\begin{center}
		\includegraphics[width=10cm]{img/app_event_input_emul.png}
		\caption{Test du driver d'entrée sur la plateforme émulée}
		\label{emul_device_input}
	\end{center}
\end{figure}


\textbf{d) Donnée: }Testez votre driver avec une application Qt disponible dans /usr/share/qt/demos. Essayez textedit
par exemple : 
\begin{lstlisting}
# export QWS_KEYBOARD="LinuxInput:/dev/input/event1"
# cd /usr/share/qt/demos/textedit
# ./textedit -qws 
\end{lstlisting}
\textbf{Travail réalisé: }
La pression sur les boutons produit des interactions avec le texte de l'application. Dans l'image ci-dessous, le texte est modifié et montre un grand espace entre le titre et le premier paragraphe.
\begin{figure}[H]
	\begin{center}
		\includegraphics[width=17cm]{img/qtextedit_test_emulator.png}
		\caption{Test du driver d'entrée sur la plateforme émulée avec application Qt QTextEdit}
		\label{qtextedit_device_input_emul}
	\end{center}
\end{figure}

\textbf{e) Donnée: }Effectuez un déploiement et un test sur la plate-forme réelle Reptar. Sur la plate-forme réelle, vous
devrez utiliser l'entrée /dev/input/event2\\\\
\textbf{Travail réalisé: }Là aussi, le texte est modifié grâce aux boutons, auxquels il a été donné des fonctionnalités telles que \textit{space}, \textit{esc}, \textit{enter} et \textit{back}.
\begin{figure}[H]
	\begin{center}
		\includegraphics[width=17cm]{img/qtextedit_on_target.jpg}
		\caption{Test du driver d'entrée sur la plateforme réelle avec application Qt QTextEdit}
		\label{qtextedit_device_input_reptar}
	\end{center}
\end{figure}

\color{red}
\textbf{Réponse aux questions: }A la fin de cette étape, vous devrez être en mesure de décrire le cheminement d'une interruption
en provenance d'un bouton jusqu'à l'effet perçu au niveau de l’application utilisateur. \\\\
\textbf{Travail réalisé: }\\\\\color{black}

\textbf{Travail supplémentaire accompli :}\\
Les interruptions provenant des boutons, dans Qemu, avaient un problème. La lecture du statut de l'interruption retournait toujours un \textit{0}, alors que l'interruption elle-même avait bien eu lieu. Il se trouve que dans le code d'émulation de la FPGA, le bit de status n'était pas mis à jour.

\begin{lstlisting}
	button_irq_register_value = button_irq_register_value | (((temp)<<1)&SP6_IRQ_BTNS_MASK);
	printf ("sp6_read: Button irq status read 0x%x (button %d made IRQ)\n",button_irq_register_value,temp+1);
	button_irq_register_value |= (1 << 4);
	return button_irq_register_value;
\end{lstlisting}
\end{document}


